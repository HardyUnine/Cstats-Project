\documentclass{article}
\usepackage{graphicx} % Required for inserting images
\usepackage{amsmath}
\usepackage{url}
\title{Midterm Report of Computational Statistics}
\author{Tanguy Godat, Tim Gouveron, Keenan Hardy}
\date{April 2025}

\begin{document}

\maketitle

\section{Abstract}
In this project, we aim to predict house prices using the Ames Housing dataset.\footnote{\url{https://www.kaggle.com/competitions/house-prices-advanced-regression-techniques/data}} 
Initially, we focus on three numerical variables---\texttt{OverallQual} (Overall Quality), \texttt{YearBuilt} (Construction Year), and \texttt{GrLivArea} (Living Area)---to explore basic relationships with \texttt{SalePrice}. Later, we plan to incorporate additional categorical features (e.g., \texttt{Neighborhood}, \texttt{ExterQual}, \texttt{KitchenQual}) using dummy encoding or ANOVA to capture their influence on house prices.  

Our approach applies classical statistical inference for exploratory analysis, followed by a regression model. Future steps involve extending the model to include categorical variables and performing more advanced statistical methods (ANOVA, time series, etc.).

\section{Introduction}

\subsection{Objective}
The goal of this project is to gain hands-on experience applying statistical techniques to a real-world dataset. Specifically, we aim to:
\begin{itemize}
    \item Explore the distribution and basic statistics of the \texttt{SalePrice} variable.
    \item Investigate correlations between \texttt{SalePrice} and selected numerical features.
    \item Eventually incorporate additional categorical features and advanced methods to improve the predictive power of our models.
\end{itemize}

\subsection{Dataset}
The Ames Housing dataset includes extensive details about house sales, such as size, quality, neighborhood, and various design features. We downloaded the dataset from Kaggle. It consists of:
\begin{itemize}
    \item \texttt{train.csv}: Features + \texttt{SalePrice} for 1{,}460 houses.
    \item \texttt{test.csv}: Features for 1{,}459 houses (no \texttt{SalePrice} column).
\end{itemize}

\section{Methods}

\subsection{Data Loading \& Preprocessing}

\begin{enumerate}
    \item \textbf{Loading the Data.} We used \texttt{pandas.read\_csv()} to load both \texttt{train.csv} and \texttt{test.csv}.
    \item \textbf{Handling Missing Values.} After checking the numerical features for missing values, we found that the three numeric features of interest (\texttt{OverallQual}, \texttt{YearBuilt}, \texttt{GrLivArea}) had no missing values. Thus, we initially focus on these complete variables.
    \item \textbf{Selecting Numeric Features for Preliminary Analysis.}
    \begin{itemize}
        \item \texttt{OverallQual}: An ordinal variable from 1--10 indicating the overall material and finish of the house.
        \item \texttt{YearBuilt}: The original construction year of the house.
        \item \texttt{GrLivArea}: Above grade (ground) living area in square feet.
        \item \texttt{SalePrice}: Our target variable.
    \end{itemize}
    \item \textbf{(Planned) Categorical Features.}
    \begin{itemize}
        \item \texttt{Neighborhood}: A categorical feature with multiple distinct neighborhoods.
        \item \texttt{ExterQual}: A categorical feature describing external quality.
        \item \texttt{KitchenQual}: A categorical feature describing kitchen quality.
    \end{itemize}
    In future steps, we intend to implement these categorical variables (possibly via dummy encoding or ANOVA) to evaluate their impact on \texttt{SalePrice}.
\end{enumerate}

\subsection{Classical Statistical Inference}

Before building any prediction models, we computed a few fundamental statistics for \texttt{SalePrice}:

\begin{itemize}
    \item \textbf{Mean, Median, and Standard Deviation} to get an initial sense of central tendency and spread.
    \item \textbf{Confidence Interval} for the mean sale price, using a t-distribution approach.
    \item \textbf{Correlation} among the three numerical predictors (\texttt{OverallQual}, \texttt{YearBuilt}, \texttt{GrLivArea}) and \texttt{SalePrice}, as a preliminary gauge of predictive strength.
\end{itemize}

\section*{Hypothesis Testing}

\subsection*{Relationship Between Overall Quality and Sale Price}

\subsubsection*{1. Introduction}
We aim to test the hypothesis that the variable \texttt{OverallQual} (Overall Quality) has an effect on \texttt{SalePrice}. Formally, our hypotheses are:
\[
\begin{aligned}
H_0 &: \beta_{\text{OverallQual}} = 0 \quad (\text{OverallQual does not affect SalePrice}),\\
H_1 &: \beta_{\text{OverallQual}} \neq 0 \quad (\text{OverallQual affects SalePrice}).
\end{aligned}
\]

\subsubsection*{2. Methods}
We fit a simple linear regression model:
\[
\text{SalePrice} = \beta_0 + \beta_1 \times (\text{OverallQual}) + \varepsilon
\]
where \(\beta_0\) is the intercept, \(\beta_1\) is the slope for \texttt{OverallQual}, and \(\varepsilon\) is the error term.

\subsubsection*{3. Results}

The regression model coefficients (estimated via R or Python) are summarized below:

\[
\begin{aligned}
\text{Intercept}:\quad & \hat{\beta}_0 = 172639.63, \\
& \text{Std. Error} = 1873.01,\quad t \approx 92.17,\quad p < 0.001, \\
& 95\%\ \text{CI} = [168965.55,\ 176313.71],\\
\\
\text{OverallQual}:\quad & \hat{\beta}_1 = 1076.57, \\
& \text{Std. Error} = 299.86,\quad t \approx 3.59,\quad p \approx 0.00034, \\
& 95\%\ \text{CI} = [488.36,\ 1664.78].\\
\end{aligned}
\]

\subsubsection*{4. Interpretation}

\begin{itemize}
    \item The intercept, \(\hat{\beta}_0\), represents the mean \texttt{SalePrice} when \texttt{OverallQual} = 0 (a necessary but not practically meaningful reference).
    \item The slope, \(\hat{\beta}_1\approx 1076.57\), indicates that for every 1-unit increase in \texttt{OverallQual}, the expected \texttt{SalePrice} increases by about \$1{,}076.57 on average.
    \item The 95\% confidence interval for \(\beta_1\) does not contain 0, suggesting a statistically significant relationship (\(p \approx 0.00034\)).
\end{itemize}

\subsubsection*{5. Conclusion}
We reject the null hypothesis \(H_0\) and conclude that \texttt{OverallQual} has a significant effect on \texttt{SalePrice}.

\vspace{1em}

\subsection*{Impact of Construction Year on Sale Price}

\subsubsection*{1. Introduction}
We aim to test whether \texttt{YearBuilt} (the year a house was constructed) has an effect on \texttt{SalePrice}. Formally:
\[
\begin{aligned}
H_0 &: \beta_{\text{YearBuilt}} = 0 \quad (\text{YearBuilt does not affect SalePrice}),\\
H_1 &: \beta_{\text{YearBuilt}} \neq 0 \quad (\text{YearBuilt affects SalePrice}).
\end{aligned}
\]

\subsubsection*{2. Methods}
We fit a simple linear regression model:
\[
\text{SalePrice} = \beta_0 + \beta_1 \times (\text{YearBuilt}) + \varepsilon,
\]
where \(\beta_0\) is the intercept, \(\beta_1\) is the slope for \texttt{YearBuilt}, and \(\varepsilon\) is the error term.

\subsubsection*{3. Results}

From R or Python output, we get:
\[
\begin{aligned}
&\hat{\beta}_0 \approx 170400,
  \quad \text{Std.\ Error} \approx 28100,
  \quad t \approx 6.07,
  \quad p < 0.001, \\
&\hat{\beta}_1 \approx 4.45,
  \quad \text{Std.\ Error} \approx 14.24,
  \quad t \approx 0.31,
  \quad p \approx 0.755, \\
& 95\%\text{ CI for } \beta_1: [-23.48,\ 32.38].
\end{aligned}
\]

\subsubsection*{4. Interpretation}

\begin{itemize}
    \item The slope coefficient, \(\hat{\beta}_1\approx 4.45\), suggests that each additional year in \texttt{YearBuilt} leads to an expected \$4.45 increase in \texttt{SalePrice}, on average, holding other factors constant.
    \item The p-value (\(\approx 0.755\)) is much larger than 0.05, so we fail to reject \(H_0\).
    \item The 95\% CI includes 0 (from -23.48 to 32.38), reinforcing the lack of evidence for a significant relationship.
    \item The coefficient of determination \(R^2\) is near 0, indicating \texttt{YearBuilt} alone explains very little variance in \texttt{SalePrice}.
\end{itemize}

\subsubsection*{5. Conclusion}
There is insufficient evidence to conclude that \texttt{YearBuilt} significantly affects \texttt{SalePrice} in a simple linear model. We fail to reject \(H_0\).

\vspace{1em}

\subsection*{Effect of Living Area on Sale Price}

\subsubsection*{1. Introduction}
We test whether the living area \texttt{GrLivArea} has a statistically significant effect on \texttt{SalePrice}. Formally:
\[
\begin{aligned}
H_0 &: \beta_{\text{GrLivArea}} = 0 \quad (\text{GrLivArea does not affect SalePrice}),\\
H_1 &: \beta_{\text{GrLivArea}} \neq 0 \quad (\text{GrLivArea affects SalePrice}).
\end{aligned}
\]

\subsubsection*{2. Methods}
A simple linear regression model is:
\[
\text{SalePrice} = \beta_0 + \beta_1 \times \text{GrLivArea} + \varepsilon.
\]
Here, \(\beta_0\) is the intercept, \(\beta_1\) is the slope for \texttt{GrLivArea}, and \(\varepsilon\) is the error term.

\subsubsection*{3. Results}
Representative output from R or Python:
\[
\begin{aligned}
& \hat{\beta}_0 \approx 150{,}500, 
  \quad \text{Std.~Error} \approx 1148, 
  \quad t \approx 131.132, 
  \quad p < 0.001, \\
& \hat{\beta}_1 \approx 19.28, 
  \quad \text{Std.~Error} \approx 0.73, 
  \quad t \approx 26.25, 
  \quad p < 0.001, \\
& \text{95\% CI for } \beta_1 \approx [17.84,\ 20.72].
\end{aligned}
\]

The coefficient of determination (\(R^2\)) is about 0.321, indicating \(\texttt{GrLivArea}\) alone explains around 32\% of the variance in \(\texttt{SalePrice}\).

\subsubsection*{4. Interpretation}
\begin{itemize}
    \item \(\hat{\beta}_1 \approx 19.28\) suggests that each additional square foot of living area increases \texttt{SalePrice} by about \$19.28 on average, holding other factors constant.
    \item The 95\% CI for \(\beta_1\) is \([17.84,\ 20.72]\), which excludes 0, indicating significance.
    \item The p-value is \(< 0.001\), so we reject \(H_0\) at conventional significance levels.
\end{itemize}

\subsubsection*{5. Conclusion}
We conclude that \texttt{GrLivArea} has a statistically significant, positive effect on \texttt{SalePrice}. We reject \(H_0\) in favor of \(H_1\), and note that the effect is practically meaningful because one additional square foot corresponds to an estimated \$19 increase in \texttt{SalePrice}.

\end{document}
